\documentclass[../main.tex]{subfiles}

\begin{document}

\chapter{C/C++高速入门}

本会快速带领大家过一遍C系的基本语法和常用特性,除了用作预习材料以外,还可以在期末考试复习的时候来快速回顾其基本语法与常用的高级特性。

这里直接从C++开始讲起,因为C++是C的超集,C的语法在C++中完全可以使用。

\section{C++的基本语法}

第一次写C++的时候,我们只需要了解一些最基本的语法和特性。记住以下三件事:

\textbf{程序有入口;先存再算;算完告诉外面。}剩下的内容,都跟我们说话一样,只不过是用C++的语法来表达,而且我们说话的句号在C++中是分号。

一个简单的C++程序如下:
\begin{verbatim}
#include <iostream>
using namespace std;

int main() {
    int age = 18;
    cout << "I am " << age; 
    return 0;
}
\end{verbatim}
逐行拆解之:
\begin{itemize}
    \item \texttt{\#include <iostream>}:告诉编译器,我要用输入输出工具。
    \item \texttt{using namespace std;}:使用标准命名空间,这样我们就可以直接使用\texttt{cout}而不需要加上\texttt{std::}前缀。
    \item \texttt{int main()}:程序的入口函数,告诉电脑程序从这里开始执行。\texttt{int}表示这个函数返回一个整数值。
    \item \texttt{int age = 18;}:跟电脑说,我要在内存找个地方放个整数,这个地方叫age,放个18进去。
    \item \texttt{cout << "I am " << age; }:把东西一股脑送送到屏幕上。
    \item \texttt{return 0;}:返回0,告诉操作系统:一切OK。
\end{itemize}

以上就是骨架。接下来该往股价里面填肉了:

\subsection{基本变量及其运算}

变量用来存储数据,可以变化;声明格式是\textbf{“先写类型,再写名字”}。

常见的变量类型有:
\begin{itemize}
    \item \texttt{int}:整数类型,通常是32位(二进制位数)。
    \item \texttt{double}:双精度浮点数,通常是64位。
    \item \texttt{char}:字符类型,通常是8位整数,表示一个字符。也可以用于表示整数。
    \item \texttt{bool}:布尔类型,表示真或假。
    \item \texttt{string}:字符串类型,表示一串字符。
\end{itemize}

对于变量的运算就跟数学差不多,比方说
\begin{verbatim}
    int a = 10;
    int b = 20;
    int c = a + b; 
    c = a * 2;
    c += 5;
\end{verbatim}
\texttt{int c = a + b}的意思是“我要创建一个变量c,把a+b的结果放进去”。可以看到,从这一行以后再提到c,就不需要再写\texttt{int}了,因为电脑已经知道c是个什么东西了。

下一行\texttt{c = a * 2}的意思是“我要把a乘以2的结果放到c里面,c以前不管是什么我都不要了”,而再下一行\texttt{c += 5}的意思是“我要把c加上5”。在上述代码中,我们发现变量c的值会随着每一行代码的执行而变化,例如第三行代码执行后,c的值变成了30;第四行代码执行后,c的值变成了20;第五行代码执行后,c的值变成了25。所以说c是一个变量。

让我们看看常见的运算符:
\begin{itemize}
    \item 四则运算:\texttt{+}(加)、\texttt{-}(减)、\texttt{*}(乘)、\texttt{/}(除)。注意,除法运算中,如果两个整数相除,结果仍然是整数,余数会被舍弃。
    \item 取模:\texttt{\%},表示取余数。例如\texttt{5 \% 2}的结果是1,因为5除以2的余数是1。
    \item 自增和自减:\texttt{++}(自增)和\texttt{--}(自减)。例如,\texttt{a++}表示将a的值加1,\texttt{b--}表示将b的值减1。
\end{itemize}
不要纠结i++和++i的区别,初学者完全可以认为这两个和\texttt{i += 1}没有区别。一个饱受诟病的题目“i = 3, i++ + i++ = ?”答:这个题目是错误的,至少是不良定义的。不同的编译器对上述代码的处理方式不同。

\textbf{我们非常不建议大家弄出这种代码:\texttt{a = b++},这种在运算或者赋值中使用自增的代码令人恼火。}如果你想要先用b的值再加1,可以写成\texttt{a = b; b++;};如果你想要先加1再用b的值,可以写成\texttt{b++; a = b;}。

更现代的程序员往往使用\texttt{i += 1}来代替\texttt{i++}。

是不是非常简单?

\subsection{判断和循环}
有时候,代码需要根据不同的条件来执行不同的操作;有时候,一段代码需要执行很多次,但是并不知道具体要执行多少次。C++提供了条件语句和循环语句来实现这些功能。

\subsubsection{条件表达式}
条件表达式是一个布尔表达式,它的值要么是true(真),要么是false(假)。在C++中,条件表达式通常用于控制程序的执行流程。一般情况下,认为false等价于0,true等价于1。

常见的一些条件表达式包括:
\begin{itemize}
    \item \texttt{==}:等于运算符。
    \item \texttt{!=}:不等于运算符。
    \item \texttt{<}:小于运算符。
    \item \texttt{>}:大于运算符。
    \item \texttt{<=}:小于等于运算符。
    \item \texttt{>=}:大于等于运算符。
    \item \texttt{\&\&}:逻辑与运算符。如果前后两个条件都为真,则结果为真;有一个是假的话,结果为假。
    \item \texttt{||}:逻辑或运算符。如果前后两个条件至少有一个为真,则结果为真;如果两个都为假,结果为假。
    \item \texttt{!}:逻辑非运算符。如果后面跟着的条件是真的,那么结果为假;反之为假
\end{itemize}



\subsubsection{条件语句}

最常见的条件语句是\texttt{if}语句。它的基本格式如下:
\begin{verbatim}
if (条件) {
    // 条件为真时执行的代码
}
else if (另一个条件) {
    // 另一个条件为真时执行的代码
}
else {
    // 以上条件全部为假时执行的代码
}
\end{verbatim}
以上代码可以有很多个\texttt{else if}分支,也可以没有\texttt{else}分支。意思是:我执行到if这一行的时候,检查后面的条件。如果条件为真,那么执行第一个大括号内的代码,剩下的全都不执行;如果条件为假,那么检查下一个\texttt{else if}的条件,如果为真就执行它的大括号内的代码,剩下的全不执行;如果所有的条件都为假,那么执行\texttt{else}大括号内的代码。

例子:
\begin{verbatim}
if (age < 18) {
    cout << "未成年";
}
else if (age < 60) {
    cout << "成年人";
}
else {
    cout << "老年人";
}
\end{verbatim}
一目了然,不言而喻。这个age变量可以是前面提到的许多类型。

\subsubsection{三元表达式}

三元表达式也是一种条件表达式,只不过它可以在一行代码中完成条件判断和结果返回,因此显得更简洁。它通常用于简单的条件判断和赋值操作。它的基本格式如下:
\begin{verbatim}
条件 ? 真值 : 假值
\end{verbatim}
以上代码的意思是:如果条件为真,整个表达式的值和真值一样;否则,整个表达式的值和假值一样。它非常适合简单的条件判断和赋值操作,但是我们不建议在复杂的条件判断中使用它或者者嵌套使用它,这样会大大降低代码的可读性。

比方说,我们可以用它来判断一个数是奇数还是偶数:
\begin{verbatim}
int n = 5;
string result = (n % 2 == 0) ? "偶数" : "奇数";
\end{verbatim}
以上代码的意思是:如果n是偶数,就把字符串“偶数”赋值给result;否则把字符串“奇数”赋值给result。

如果使用

\subsubsection{切换语句}
有时候,我们需要根据一个变量的值来执行不同的操作。C++提供了\texttt{switch}语句来实现这个功能。它的基本格式如下:
\begin{verbatim}
switch (变量) {
    case 值1:
        // 当变量等于值1时执行的代码
        break;
    case 值2:
        // 当变量等于值2时执行的代码
        break;
    default:
        // 当变量不等于任何case的值时执行的代码
}
\end{verbatim}
以上代码的意思是:检查变量的值,如果等于值1,就执行第一个大括号内的代码;如果等于值2,就执行第二个大括号内的代码;如果都不等于,就执行\texttt{default}大括号内的代码。可以有任意多个\texttt{case}分支,也可以没有\texttt{default}分支。

注意,\texttt{break}语句用于跳出\texttt{switch},这个是必须的。

例子:
\begin{verbatim}
switch (day) {
    case 1:
        cout << "星期一";
        break;
    case 2:
        cout << "星期二";
        break;
    case 3:
        cout << "星期三";
        break;
    // ......其他的,基本一个写法
}
\end{verbatim}
这也一目了然不言而喻了。

\subsubsection{for循环语句}

for循环是灵活度极高的循环语句。它的基本格式如下:
\begin{verbatim}
for (初始化; 条件; 更新) {
    // 循环体
}
\end{verbatim}
以上代码的意思是:先执行初始化语句,然后检查条件是否为真。如果为真,就执行循环体内的代码,然后执行更新语句。接着再检查条件,如果为真就继续执行循环体,否则跳出循环。比方说,我们需要打印1到10的数字,可以这样写:
\begin{verbatim}
for (int i = 1; i <= 10; i++) {
    cout << i << " ";
}
\end{verbatim}
这段代码的意思是:先初始化一个变量i为1,然后检查i是否小于等于10。如果是,就打印i的值,然后将i加1。接着再检查i是否小于等于10,如果是就继续打印,否则跳出循环。

这个\texttt{int i = 1}可以在其他地方声明过,那么这里就遵从“先声明后使用”的原则,不需要再写int了。

\subsubsection{while和do-while循环语句}
while循环是另一种常见的循环语句。它的基本格式如下:
\begin{verbatim}
while (条件) {
    // 循环体
}
\end{verbatim}
以上代码的意思是:先检查条件是否为真。如果为真,就执行循环体内的代码,然后再次检查条件。如果条件仍然为真,就继续执行循环体,否则跳出循环。比方说,我们需要打印1到10的数字,可以这样写:
\begin{verbatim}
int i = 1;
while (i <= 10) {
    cout << i << " ";
    i++;
}
\end{verbatim}
以上内容与for循环的例子是等价的。

do-while循环与while循环类似,但它会先执行一次循环体,然后再检查条件。它的基本格式如下:
\begin{verbatim}
do {
    // 循环体
} while (条件);
\end{verbatim}
这样可以保证循环体至少执行一次。比方说,我们需要打印1到10的数字,可以这样写:
\begin{verbatim}
int i = 1;
do {
    cout << i << " ";
    i++;
} while (i <= 10);
\end{verbatim}

这些看起来都非常简单。而以上内容就是C++的全部基本语法了:是的,你已经学完了!

让我们来做个小练:

\subsection{基本语法小练}

角谷猜想是一个有趣的数学问题:从任意整数开始,如果他是奇数就乘以3加1,如果是偶数就除以2,如此反复循环,最终一定会得到1。目前还没有人证明这个猜想,但我们可以用C++来验证一些具体值。

请编写一个C++程序,输入一个整数n,然后输出这个整数经过角谷猜想的处理后,最终得到1的过程。要求输出每一步的结果。
例如,输入n=6时,输出应该是:6, 3, 10, 5, 16, 8, 4, 2, 1。

\subsubsection{边写边说}

一句一句地看:任意整数,奇数乘以3加1,偶数除以2。这一段代码写起来很方便。我们知道,整数的奇偶性可以通过取模来判断:如果n\%2==0,那么n是偶数;否则n是奇数。

因此仅仅这句话,从人的自然语言到代码语言的转换就非常简单了。
\begin{verbatim}
int n;
if(n % 2 == 0) {
    n /= 2; // 偶数除以2
} else {
    n = n * 3 + 1; // 奇数乘以3加1
} 
\end{verbatim}

或者使用三元表达式:
\begin{verbatim}
n = (n % 2 == 0) ? (n / 2) : (n * 3 + 1);
\end{verbatim}

然后是下一句话:如此反复循环。这说明我们至少需要写一个循环语句。再看下一句:最终一定会得到1。

这样我们就明确了:我们需要写一个循环语句,跳出循环的条件是n等于1。于是我们可以写成:
\begin{verbatim}
while(n != 1){
    // 处理 n 的代码
}
\end{verbatim}

再看下一句:输入一个整数n,输出每一步的结果。这说明我们需要一个输入语句和一个输出语句。输入语句可以用\texttt{cin},输出语句可以用\texttt{cout}。

题目读完了,那么我们就可以把这些代码组合起来了:
\begin{verbatim}
int n;
cin >> n; // 输入一个整数n
cout << n; // 输出初始值
while(n != 1) {
    if(n % 2 == 0) {
        n /= 2; // 偶数除以2
    } else {
        n = n * 3 + 1; // 奇数乘以3加1
    }
    cout << " " << n; // 输出每一步处理后的结果
}
\end{verbatim}
这就是基本的代码框架。下一步,我们结合一开始说的话:程序要有入口,先存再算,算完告诉外面。于是我们可以真正地完成这段代码:
\begin{verbatim}
#include <iostream>
using namespace std;

int main() {
    int n;
    cin >> n; // 输入一个整数n
    cout << n; // 输出初始值
    while(n != 1) {
        if(n % 2 == 0) {
            n /= 2; // 偶数除以2
        } else {
            n = n * 3 + 1; // 奇数乘以3加1
        }
        cout << " " << n; // 输出每一步处理后的结果
    }
    return 0; // 返回0,表示程序正常结束
}
\end{verbatim}
这段代码就是一个完整的C++程序了。同学们可以在自己的电脑上编译运行,看看效果!

非常简单,对不对?下面用三元表达式来改写一下这个程序试试看!

\section{C++的进阶使用}

\subsection{更进阶的变量类型}
C++提供了许多更进阶的变量类型和特性,可以帮助我们更好地组织代码和数据。以下是一些常见的进阶变量类型和特性:
\subsubsection{数组}
数组是一个可以存储多个同类型数据的变量。它的基本格式如下:
\begin{verbatim}
类型 数组名[大小];
\end{verbatim}
以上代码的意思是:声明一个名为数组名的数组,它可以存储大小个同类型的数据。数组的索引从0开始。例如,我们可以声明一个整数数组来存储10个整数:
\begin{verbatim}
int arr[10];
\end{verbatim}
以上的数组arr中的元素可以通过索引来访问,例如arr[0]表示第一个元素,arr[1]表示第二个元素,以此类推,一直到arr[9]表示第十个元素。没有arr[10],访问这个会出错。

\subsubsection{字符串}
C++系的字符串类型是\texttt{string},它可以存储一串字符。字符串的基本格式如下:
\begin{verbatim}
#include <string>   // 引入字符串库
string str = "Hello, World!";
\end{verbatim}
引用字符串库是必要的,否则编译器可能会报错;这个库还提供了一些对字符串进行操作的方法,非常方便。

字符串的本质是一个数组,存储了一串字符(C风格的字符串正是char[])。我们可以通过索引来访问字符串中的字符,例如str[0]表示第一个字符,str[1]表示第二个字符,以此类推。

字符串的长度可以通过\texttt{str.length()}方法来获取。除此以外,还有很多字符串操作方法,例如\texttt{str.substr()}(获取子串)、\texttt{str.find()}(查找子串)等。

\subsubsection{结构体}
结构体是一个可以存储多个不同类型数据的变量。它的基本格式如下:
\begin{verbatim}
struct 结构体名 {
    类型 成员名1;
    类型 成员名2;
    // ...
};
\end{verbatim}
以上代码的意思是:声明一个名为结构体名的结构体,它可以存储多个不同类型的数据。结构体的成员可以是任意类型,包括基本类型、数组、字符串等。
例如,我们可以声明一个表示学生的结构体:
\begin{verbatim}
struct Student {
    string name;  // 学生姓名
    int age;      // 学生年龄
    double gpa;   // 学生绩点
};

Student student1;  // 声明一个学生变量
student1.name = "Alice";  // 设置学生姓名
student1.age = 20;  // 设置学生年龄
student1.gpa = 3.5;  // 设置学生绩点
cout << "Name: " << student1.name << ", Age: " << student1.age << ", GPA: " << student1.gpa << endl;
\end{verbatim}

以上内容很好地展示了怎么定义、声明、使用一个结构体。结构体的成员可以通过点(.)运算符来访问,例如\texttt{student1.name}表示学生1的姓名。

结构体可以帮助我们更好地组织数据,使得代码更易读。

\subsection{函数和作用域}

有时候,我们需要在这个地方使用一些代码,在另外一个地方也使用同样的代码。为了避免重复编写代码,我们可以将这些代码封装成一个函数。函数是一个可以重复调用的代码块,它可以接受参数并返回结果。

函数的基本格式如下:
\begin{verbatim}
返回类型 函数名(参数列表) {
    // 函数体
    return 返回值;  // 如果返回类型不是void,则需要返回一个值
}
\end{verbatim}
以上代码的意思是:声明一个名为函数名的函数,它可以接受参数列表中的参数,并返回一个返回类型的值。函数体是函数的具体实现。

例如,我们可以声明一个计算两个整数和的函数:
\begin{verbatim}
int add(int a, int b) {
    return a + b;  // 返回a和b的和
}
\end{verbatim}
我们可以在其他函数中调用这个函数:
\begin{verbatim}
int main() {
    int x = 5;
    int y = 10;
    int sum = add(x, y);  // 调用add函数
    cout << "Sum: " << sum << endl;  // 输出结果
    return 0;
}
\end{verbatim}



\end{document}