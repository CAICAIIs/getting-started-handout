\documentclass[../main]{subfiles}

\begin{document}

\chapter{C系包管理}

在去年(2024)学年初期,我被我某节课的助教邀请,帮助他的恋人配置C++开发环境与调包。当时我调了一晚上也没调好,彻底意识到为C++调包是极为痛苦的体验:C++包管理做得很烂,和Python的包管理难度相比,简直是一个天上一个地下。

最近,这个问题再度被同学们提出。我认为,在很多课程上使用C++编写项目的时候,也逃不开使用包(但是很多课程并不提及怎么装包);为了防止同学们在使用C++包时遇到不必要的麻烦,我决定写一章来介绍C++的包管理。

当然,每一个包都有着自己推荐的安装方式。如果官方文档等有自己推荐的安装方式,建议同学们优先使用官方推荐的方式安装包。另一方面,C++的上限极高,但是下限也极低,很多包的文档质量极差,甚至没有文档;因此,本文也会介绍一些通用的安装方式。(当然这种包我们还是不推荐使用的。)

\emph{实际上C++包管理的最好方式是:USE RUST INSTEAD!(大嘘)}

\section{C++包管理的困境和现状}

虽然我们C++有一定的标准协会(ISO C++),但是C++没有官方的运行时(Runtime)。这导致C++的ABI随着编译器、版本、编译选项而随地大小变,导致二进制包必须严丝合缝。

简单说来,C++装库主要任务有三项:找到头文件、找到库文件、让构建系统知道它们在哪。头文件指的是C++的头文件(.h、.hpp等),库文件指的是编译好的二进制文件(.so、.dll、.a等)。由于二十多年来没有一个官方组织来统一C++的包管理、大家习惯自己编译,因此逐渐形成了两种互相不重叠的技术路线:
\begin{itemize}
    \item 手动装
    \item 使用包管理器
\end{itemize}

\section{手动装}

手动装是最原始的方式,要么从源代码编译,要么从二进制包安装。手动装的好处是可以完全控制安装的版本和配置,坏处是需要手动处理依赖关系和路径问题,并且容易出错。不过,其他的方式最终还是依赖于手动装的过程,只不过脏活累活有自动化工具帮你做了;另一方面,手动装包也是最通用的方式,在包管理器上没有找到包时,手动装包几乎是唯一的选择。

\subsection{手动编译}

手动编译的过程通常是:先从官网下载源代码,然后编译:
\begin{lstlisting}
git clone https://github.com/fmtlib/fmt.git
cmaks -S fmt -B build -DCMAKE_BUILD_TYPE=Release -DCMAKE_POSITION_INDEPENDENT_CODE=ON
cmake --build build
sudo cmake --install build --prefix /opt/fmt
\end{lstlisting}

然后使用CMake配置编译器的搜索路径:
\begin{lstlisting}
    find_package(fmt REQUIRED)
    target_link_libraries(my_program PRIVATE fmt::fmt)
\end{lstlisting}

完毕。以上方式是一种最通用的手动安装方式之一,也很容易适用于其他包。需要注意的是,CMake的find\_package命令会在系统的标准路径下查找包,因此如果你将包安装在非标准路径下,需要手动指定搜索路径等。

\subsection{头文件库}

这种安装方式适宜用于一些小型的、单文件的库。只需要将头文件放在项目目录(\texttt{third\_party/})下,然后在CMake中添加头文件搜索路径即可:
\begin{lstlisting}
    add_subdirectory(third_party/nlohmann_json)
    target_include_directories(my_program PRIVATE third_party/nlohmann_json)
\end{lstlisting}

完毕。以上方式是“把依赖当源码”的极限版本,没有ABI问题,但是需要手动处理依赖关系和版本问题等。

\section{使用包管理器}

包管理器实际上是一种自动化的手动装方式。它们通常会提供一个统一的命令行工具,来管理包的安装、卸载、更新等操作。许多包都可以使用这种方式安装,且包管理器通常会处理依赖关系和路径问题。

\subsection{系统级包管理器}

对于特定的操作系统或开发环境,使用其自带的包管理器是安装C++库最直接的方式。无论是Linux(如APT、DNF、Pacman)、macOS(如Homebrew),还是 Windows上的MSYS2环境都提供了强大的包管理工具。当你的开发环境和目标部署环境一致时,系统级包管理器通常是最佳选择。

以安装\texttt{fmt}库为例,不同平台上的命令如下:

\begin{itemize}
    \item \textbf{Debian/Ubuntu (APT)}:
    \begin{lstlisting}[language=bash]
sudo apt install libfmt-dev
    \end{lstlisting}
    \item \textbf{Fedora (DNF/YUM)}:
    \begin{lstlisting}[language=bash]
sudo dnf install fmt-devel
    \end{lstlisting}
    \item \textbf{Arch Linux (Pacman)}:
    \begin{lstlisting}[language=bash]
sudo pacman -S fmt
    \end{lstlisting}
    \item \textbf{macOS (Homebrew)}:
    \begin{lstlisting}[language=bash]
brew install fmt
    \end{lstlisting}
    \item \textbf{Windows (MSYS2/MinGW)}: 在 MSYS2 环境下,同样可以使用 \texttt{pacman}。
    \begin{lstlisting}[language=bash]
pacman -S mingw-w64-ucrt-x86_64-fmt
    \end{lstlisting}
\end{itemize}

安装后,这些库通常会被放置在系统的标准路径下(如\texttt{/usr/include}和\texttt{/usr/lib})。构建系统(如CMake等)通过\texttt{find\_package}命令可以轻松地找到它们,无需额外配置:
\begin{lstlisting}
find_package(fmt REQUIRED)
target_link_libraries(my_program PRIVATE fmt::fmt)
\end{lstlisting}

\subsubsection{优势}
\begin{itemize}
    \item 一条命令即可完成安装,自动处理复杂的依赖关系。
    \item 库文件和头文件被安装到标准位置,构建系统可以自动发现,无需手动配置路径。
    \item 可以与其他系统软件一同通过包管理器进行更新和维护。
    \item 发行版仓库中的库经过测试,通常与系统中的其他组件(如编译器)兼容性良好。
\end{itemize}

\subsubsection{劣势}
\begin{itemize}
    \item 仓库中的库版本未必能满足所有需求。如果你的项目需要某个库的最新特性,或者依赖于一个非常特定的旧版本,某些包管理器可能无法满足需求。
    \item 不同平台、不同发行版的包管理器命令和包名各不相同,往往需要手动核验与对应。
    \item 不是所有的 C++ 库都被收录到了官方仓库中,尤其是那些比较小众或新兴的库。
\end{itemize}

\subsection{vcpkg}

vcpkg是微软开发的C++包管理器。我们建议在Windows使用这个包管理器,因为它可以很好地与Visual Studio集成。vcpkg的使用方式非常简单,首先应当安装vcpkg(过程略)。

然后在项目中添加vcpkg的CMake配置:(你应当把示例路径替换为你自己安装vcpkg的实际路径)
\begin{lstlisting}
    cmake -S . -B build -DCMAKE_TOOLCHAIN_FILE=/path/to/vcpkg/scripts/buildsystems/vcpkg.cmake
\end{lstlisting}
下一步声明依赖\texttt{vcpkg.json}文件:
\begin{lstlisting}
    {
        "name": "demo",
        "version": "0.1.0",
        "dependencies": [
            "fmt",
            "nlohmann-json"
        ]
    }
\end{lstlisting}
这样,在CMake构建文件的时候就会自动下载源码并编译出*.a或者*.so等文件,并且将它们放在build的相关目录下。

vcpkg的好处是源码优先(可以切换triplet)、不和系统冲突;坏处是仅能支持CMake项目。

\subsection{Conan}

Conan是一个跨平台的C++包管理器,支持多种构建系统,包括CMake、Makefile等。Conan的使用方式也很简单:
\begin{lstlisting}
    pip install conan # 安装Conan
    conan profile detect --force # 检测当前系统配置并生成配置文件
\end{lstlisting}
然后在项目中添加Conan的配置文件(conanfile.txt):
\begin{lstlisting}
    [requires]
    fmt/9.0.0
    nlohmann_json/3.11.2

    [generators]
    CMakeDeps
    CMakeToolchain
\end{lstlisting}
然后构建:
\begin{lstlisting}
    conan install . --build=missing -s build_type=Release
    cmake -S . -B build -DCMAKE_TOOLCHAIN_FILE=build/conan_toolchain.cmake
\end{lstlisting}

Conan的好处是支持多种构建系统和平台,包括makefile、meson、bazel等;二进制包仓库丰富;支持版本锁定等。但是缺点是conanfile和CMakeLists.txt文件需要手动维护和同步买比较麻烦。

\subsection{conda-forge}

conda-forge也能管理C++包:
\begin{lstlisting}
    conda install -c conda-forge fmt nlohmann_json
\end{lstlisting}
当然conda做这个还是有些大材小用了。

\end{document}