\documentclass[../main.tex]{subfiles}

\begin{document}
\chapter{常用构建工具链概览}

讲完了开发的各个方面,现在该讲讲怎么把你的代码构建成可执行的文件(或包),以便于之后的发布和使用。

一个比较古老的构建方法是使用\texttt{makefile}构建,这是非常经典的构建方式,几乎所有的编程语言都可以使用。但是因为风格过老,现在已经不大流行。目前较为常见的构建方式是\texttt{CMake}和\texttt{XMake},它们风格现代,功能强大。

\section{CMake}

C系语言功能非常强大是公认的不争事实。但是它的构建一直是一个巨大的痛点:使用include来链接库是非常麻烦的事情,文件一多,gcc直接变成一种折磨(\texttt{gcc a.c b.c ... -o project},还得管依赖关系,还得管编译顺序),困难得让人完全不想用。

在2000年左右,CMake作为一个跨平台的构建工具被提出。它的主要目标是提供一个简单的方式来管理大型项目的构建过程。在当时,autotools难写,Makefile难以跨平台,而CMake“描述意图”代替“描述过程”,使得构建过程变得更简单。

好风凭借力。随着CMake渐渐进入人们的视野,KDE、LLVM、OpenCV、Qt6、ROS2等项目全都开始使用CMake作为构建工具,社区生态空前繁荣,直接将CMake推上了构建工具的顶峰。直到现在,CMake依然是面向生产环境的“事实标准”,VS、CLion、QtCreator、VS Code全部将CMake作为一级公民,GH Actions、Docker、Conan、vcpkg无需装包,默认识别CMake项目。

因此,不学会CMake,你在今日的开发世界中寸步难行:除非你愿意用Makefile这种玩意为每个IDE和平台写一大堆构建脚本!

\subsection{最小运行实例}

CMake的文件名称默认是\texttt{CMakeLists.txt}。除非特殊情况,以后不再提及文件名称。

\subsubsection{单一文件}

假设现在存在一个\texttt{main.cpp}文件(不管内容是什么了)。在同一目录下新建一个CMake文件,内容如下:

\begin{lstlisting}
cmake_minimum_required(VERSION 3.10) # 声明最低CMake版本
project(hello LANGUAGES CXX) # 声明项目名称和使用的语言
add_executable(hello main.cpp) # 声明可执行文件hello,源文件为main.cpp
\end{lstlisting}

然后构建:
\begin{lstlisting}[language=bash]
    cmake -B build # 生成构建系统
    cmake --build build # 构建
    ./build/hello # 运行可执行文件(Linux和mac)
    ./build/Debug/hello.exe # 运行可执行文件(Windows)
\end{lstlisting}

爽。

\subsubsection{多个目录}

假设现在有一个目录结构:
\begin{lstlisting}
src/
- main.cpp
- math/
    - add.cpp
    - add.hpp
- io/
    - print.cpp
    - print.hpp
\end{lstlisting}

根目录CMake文件如下:
\begin{lstlisting}
cmake_minimum_required(VERSION 3.10)
project(multidir)

add_subdirectory(src) # 告诉CMake:src里面还有东西
\end{lstlisting}

src/CMake文件如下:
\begin{lstlisting}
    add_library(math STATIC math/add.cpp) # 声明静态库math,源文件为add.cpp
    add_library(io STATIC io/print.cpp) # 声明静态库io,源文件为print.cpp
    add_executable(app main.cpp) # 声明可执行文件app,源文件为main.cpp
    target_link_libraries(app PRIVATE math io) # 将math和io库链接到app
\end{lstlisting}

构建同单文件一样,不打第二遍。

我们看到一个问题:CMake需要每个目录一个,职责清晰;同时,库和可执行文件都是“目标”。目标这个词在CMake中非常重要,几乎所有的操作都是针对目标进行的,后文会反复出现。

\subsection{语法}

\subsubsection{目标和作用域}

现代的CMake的语法基于目标,把属性贴在目标上,谁链接谁可见。比如说:
\begin{lstlisting}
add_library(foo STATIC foo.cpp) # 告诉CMake,这有个库foo,是静态的
target_include_directories(foo PUBLIC include) # foo库的头文件在include目录下
target_compile_features(foo PUBLIC cxx_std_20) # foo库需要C++20特性
target_link_libraries(foo PUBLIC bar) # foo库需要链接bar库
\end{lstlisting}

这里的\texttt{PUBLIC}是一个作用域,表示这个属性对链接到foo的目标可见。作用域有三个:
\begin{itemize}
    \item \texttt{PRIVATE}:给自己用
    \item \texttt{INTERFACE}:给下游用,自己不用
    \item \texttt{PUBLIC}:自己和下游都用
\end{itemize}

一般情况下,除非显式传递,否则子目录自动继承父目录作用域;而对于一个特定的目标,属性自动跟随,跨目录也能传递。类似\texttt{link\_directories}这样的命令是全局的命令,在现代CMake中非常不推荐使用。

\subsubsection{变量、缓存、生成器}

对于CMake而言,我们可以声明一些变量,使得CMake源码更简洁。例如:
\begin{lstlisting}
    set(SOURCE_FILES main.cpp math/add.cpp io/print.cpp) # 声明变量SOURCE_FILES
    add_executable(app ${SOURCE_FILES}) # 使用变量
\end{lstlisting}
这样可以省去很多重复的代码。

CMake的变量分为两种:缓存变量和普通变量。普通变量用户是修改不了的,而缓存变量可以通过命令行或CMake GUI修改。缓存变量通常用于配置选项,比如:
\begin{lstlisting}
option(BUILD_TESTS "Build tests" ON) # 声明一个缓存变量BUILD_TESTS,默认值为ON
set(CMAKE_BUILD_TYPE "Release" CACHE STRING "Build type") # 声明一个缓存变量CMAKE_BUILD_TYPE,默认值为Release
\end{lstlisting}

用户可以修改这些缓存变量的值,以定制构建过程。
\begin{lstlisting}[language=bash]
cmake -B build -DBUILD_TESTS=OFF -DCMAKE_BUILD_TYPE=Debug # 修改缓存变量,将BUILD_TESTS设置为OFF,将CMAKE_BUILD_TYPE设置为Debug
\end{lstlisting}

生成器决定了CMake生成的构建系统类型。CMake支持多种生成器,比如Makefile、Ninja、Visual Studio等。可以通过\texttt{-G}选项指定生成器。目前较为常见的生成器有:

\begin{itemize}
    \item \texttt{Ninja}:一个快速的构建系统,CMake默认生成的构建系统。跨平台最快。
    \item \texttt{Unix Makefiles}:传统的Makefile生成器,Linux服务器默认。
    \item \texttt{Visual Studio}:Windows下的Visual Studio生成器。
    \item \texttt{Xcode}:macOS下的Xcode生成器。
\end{itemize}
使用这些生成器非常简单:
\begin{lstlisting}[language=bash]
cmake -B build -G Ninja # 使用Ninja生成器
\end{lstlisting}

\subsubsection{条件判断、执行命令、输出信息}

在CMake中,我们可以使用条件判断来控制构建过程。这在使用了缓存变量的时候非常有用:
\begin{lstlisting}
if(BUILD_TESTS) # 如果BUILD_TESTS为真
    enable_testing() # 启用测试
    add_subdirectory(tests) # 添加tests目录
endif() # 结束条件判断
\end{lstlisting}

这个if的条件写法和C++的if类似,也可以写\texttt{elseif()}(没有空格)和\texttt{else()},含义也相同;但是不支持运算符。所有的运算符都应该用字符串表示:
\begin{itemize}
    \item AND、OR、NOT:御三家,不用解释都知道这是什么
    \item EQUAL、LESS、GREATER:比较运算符,分别表示等于、小于、大于
    \item EXIST:后面的东西(绝对路径)若存在,则为真;否则为假
    \item DEFINED:后面的变量若已定义,则为真;否则为假
    \item COMMAND:后面的命令(宏、函数)若存在并能执行,则为真;否则为假
    \item STREQUAL、STRLESS、STRGREATER:字符串比较运算符,分别表示等于、小于、大于。使用前提是字符串必须是有效的数字。
\end{itemize}

我们可以使用\texttt{execute\_process}命令来执行外部命令:
\begin{lstlisting}
execute_process(COMMAND echo "Hello, World!" OUTPUT_VARIABLE output RESULT_VARIABLE result) # 执行命令,输出到output变量,结果状态码到result变量
\end{lstlisting}
一次可以同时执行许多命令,这些命令是并行的;每一个子进程的标准输出都会映射到下一个进程的标准输入;所有的子进程公用一个标准错误输出管道。除了COMMAND外,其余关键字均可以省略。

有时候,我们需要输出一些信息到控制台上(例如错误信息等)。CMake提供了\texttt{message}命令来输出信息:
\begin{lstlisting}
message(STATUS "This is a status message") # 输出状态信息
message(WARNING "This is a warning message") # 输出警告信息
message(ERROR "This is an error message") # 输出错误信息
message(FATAL_ERROR "This is a fatal error message") # 输出致命错误信息并终止构建
\end{lstlisting}

\subsubsection{列表}

CMake支持列表(list)。列表一般需要使用set来定义。以下是一些常用的列表操作命令:
\begin{lstlisting}
set(SOURCES main.cpp math/add.cpp io/print.cpp) # 定义一个列表SOURCES
list(APPEND SOURCES io/scan.cpp) # 向列表SOURCES添加一个元素
list(REMOVE_ITEM SOURCES io/scan.cpp) # 从列表SOURCES中移除一个元素
list(LENGTH SOURCES length) # 获取列表SOURCES的长度,存储到length
list(GET SOURCES 0 first) # 获取列表SOURCES的第一个元素,存储到first
list(SORT SOURCES) # 对列表SOURCES进行排序
list(FIND SOURCES main.cpp index) # 查找列表SOURCES中元素main.cpp的位置,存储到index
list(INSERT SOURCES 0 "io/scan.cpp") # 在列表SOURCES的开头插入元素io/scan.cpp
list(JOIN SOURCES ", " joined) # 将列表SOURCES用逗号和空格连接成一个字符串,存储到joined
\end{lstlisting}

基本上是所见即所得。可以看到,列表操作和C++ STL的vector类似。

\subsubsection{调库}

有时候我们需要调外部库,例如OpenCV等。我们把调库分为两种:系统库和第三方库。

如果系统已经安装,使用\texttt{find\_package}命令即可:
\begin{lstlisting}
find_package(fmt REQUIRED) # 查找fmt库,REQUIRED表示必须找到
target_link_libraries(app PRIVATE fmt::fmt) # 将fmt库链接到app
\end{lstlisting}
上述提到的\texttt{fmt}库是一个常用的C++格式化库,Linux发行版可以通过包管理器安装(例如Ubuntu的\texttt{apt install libfmt-dev})。

如果系统没安装,则需要使用FetchContent来当场下载。FetchContent是CMake 3.11引入的模块。示例代码如下:
\begin{lstlisting}
include(FetchContent) # 引入FetchContent模块
FetchContent_Declare(
    googletest
    GIT_REPOSITORY https://github.com/google/googletest.git
    GIT_TAG ? # 指定版本,这里没有指定,实际应当指定一个
)
FetchContent_MakeAvailable(googletest) # 下载并编译googletest库
\end{lstlisting}
FetchContent会自动下载并编译googletest库,并将其链接到当前项目。这样构建的好处是不需要预装库,直接构建时下载并编译,适合CI/CD等场景。缺点:首次编译时间长。

除此之外,CMake还支持使用Conan、vcpkg等包管理器来管理第三方库。Conan和vcpkg都是非常流行的C++包管理器,可以自动处理依赖关系和版本问题。
\begin{lstlisting}
    vcpkg install fmt # 使用vcpkg安装fmt库
    cmake -B build -DCMAKE_TOOLCHAIN_FILE=$VCPKG_ROOT/scripts/buildsystems/vcpkg.cmake # 使用vcpkg的CMake工具链文件
\end{lstlisting}
vcpkg会将库链接到当前项目,例如上文会将fmt::fmt暴露给\texttt{find\_package}。

\subsection{工具链}

工具链文件(ToolChain文件)是一段纯粹的CMake脚本,在project命令之前被CMake加载,用于通知当前线索的重要信息:目标平台是什么?使用什么交叉编译器?头文件什么的在哪?默认编译、链接flags是什么?

闲的没事的人可以使用cmake -D一个个传。这么做最大的问题是:太多了,先不说传错、传漏的可能性,光说脚本的可读性就够我们喝一壶了。

假设我们现在希望在Linux机器上编译一个Win32程序,那么工具链文件可以是这样的:
\begin{lstlisting}
# 文件名:cross-mingw.cmake
# 1. 目标系统
set(CMAKE_SYSTEM_NAME Windows)        # 告诉 CMake “我要生成 Windows PE”
set(CMAKE_SYSTEM_PROCESSOR i686)      # 目标 CPU

# 2. 交叉编译器
set(CMAKE_C_COMPILER   i686-w64-mingw32-gcc)
set(CMAKE_CXX_COMPILER i686-w64-mingw32-g++)

# 3. sysroot / 搜索根目录
set(CMAKE_FIND_ROOT_PATH /usr/i686-w64-mingw32)

# 4. 查找策略:头文件/库只在目标环境里找,可执行程序在宿主环境里找
set(CMAKE_FIND_ROOT_PATH_MODE_PROGRAM NEVER)
set(CMAKE_FIND_ROOT_PATH_MODE_LIBRARY ONLY)
set(CMAKE_FIND_ROOT_PATH_MODE_INCLUDE ONLY)
\end{lstlisting}    
之后编译:
\begin{lstlisting}[language=bash]
cmake -B build-win -DCMAKE_TOOLCHAIN_FILE=cross-mingw.cmake
cmake --build build-win
./build-win/hello.exe # 运行可执行文件
\end{lstlisting}
上述代码会在Linux上编译一个Win32程序,但是可以直接拷贝到Windows上运行。

当然,这只是一个最基本的示例,实际上工具链的使用相当复杂,甚至超过了CMake本身的复杂度。但是,交叉编译的脏活累活全是它做的,正是工具链的使得我们无需使用一大堆\texttt{-D}参数来传递信息。因此将来如果有的同学有志于从事嵌入式开发等工作,建议还是去官方网站上学习一下工具链的使用。

\subsection{安装、导出、打包}

CMake支持安装、导出和打包功能。安装功能可以将构建好的目标安装到指定目录,导出功能可以将目标导出为一个CMake包,打包功能可以将目标打包为一个可分发的文件。

安装功能使用\texttt{install}命令:
\begin{lstlisting}
install(TARGETS foo EXPORT fooTargets # 安装foo目标,并导出为fooTargets
    RUNTIME DESTINATION bin # 可执行文件安装到bin目录
    LIBRARY DESTINATION lib # 动态库安装到lib目录
    ARCHIVE DESTINATION lib # 静态库安装到lib目录
)
install(DIRECTORY include/ DESTINATION include) # 头文件安装到include目录
\end{lstlisting}

导出功能使用\texttt{export}命令:
\begin{lstlisting}
install(EXPORT fooTargets
    FILE foo-config.cmake
    NAMESPACE foo::
    DESTINATION lib/cmake/foo) # 导出fooTargets为foo-config.cmake文件,命名空间为foo::
\end{lstlisting}
导出后下游项目可以直接\texttt{find\_package(foo REQUIRED)}来使用foo库。

打包功能使用\texttt{cpack}命令:
\begin{lstlisting}
set(CPACK_GENERATOR "DEB")
set(CPACK_PACKAGE_NAME "mylib")
include(CPack)
\end{lstlisting}
上述代码会生成一个DEB包,包名为mylib。构建的时候,使用以下命令可以得到打包文件:
\begin{lstlisting}[language=bash]
cmake --build build --target package # 构建并打包
\end{lstlisting}

\subsection{常见坑}
CMake虽然强大,但也有一些常见的坑需要注意:
\begin{itemize}
    \item 缓存残留:目录改了名,CMake仍然记得旧值,这时候把整个build目录全删了就可以了。
    \item 找不到头文件:看看target\_include\_directories是否正确设置了作用域,八成是错写成了\texttt{PRIVATE}。
    \item 找不到dll(Windows特供):把bin目录加入PATH,或者运行以下命令后运行\texttt{install/ bin}下的可执行文件:
\begin{lstlisting}
    cmake --install build --prefix install
\end{lstlisting}
    \item 生成器表达式看不懂:在在 CMakeLists 里 \texttt{message(STATUS "\$<TARGET\_FILE:foo>")} 打印调试。
\end{itemize}

以上内容仅仅是给CMake的一个概览。虽然可以让同学们理解CMake的基本用法,但要深入使用CMake,还是需要阅读官方文档和相关教程。毕竟CMake的命令非常多(就一个\texttt{execute\_process}就有一大堆选项,合起来能占满半个屏幕),而且每个命令的用法也非常灵活。

这里帮各位把\href{https://cmake.org/cmake/help/latest/guide/tutorial/}{CMake官方说明文档}贴出来了,感兴趣的同学可以去看看;也可以看看这本电子书:Professional CMake: A Practical Guide(作者:Craig Scott),这本书是CMake的权威指南,内容非常全面,适合深入学习CMake。

\section{XMake}

不少同学可能会疑惑:怎么又来一个XMake?

这是因为,CMake虽然已经很好了,但是语法还是有些复杂,有点像“外星语”。而XMake比CMake现代的多,语法也更简单,使用lua脚本而不是CMake特有的脚本,更容易上手。同时,XMake把CMake的configure、generate、build等步骤压成一步,且自带包管理,非常舒适。当然,也因为它太新了,所以大多数情况下使用CMake已经可以满足需求了;不过在这里我还是简单讲一下吧,毕竟PKU的部分课程提供的引擎需要我们手写XMake脚本。

安装非常简单,包管理器直接解决,不管用的winget还是apt,都可以直接安装XMake:
\begin{lstlisting}[language=bash]
winget install xmake # Windows
apt install xmake # Ubuntu
brew install xmake # macOS
\end{lstlisting}

\subsection{最小运行实例}

XMake的构建文件名称默认是\texttt{xmake.lua},本章不再赘述。

运行以下命令看看模板:
\begin{lstlisting}
xmake create -l c++ hello # 创建一个C++项目hello
xmake create -l c++ -t static hello_lib # 创建一个C++静态库项目hello_lib
\end{lstlisting}

我们发现当前目录多了个hello文件夹,结构是这样的:
\begin{lstlisting}
hello/
- xmake.lua # 构建文件
- src/
    - main.cpp # 源文件
\end{lstlisting}

如果希望构建并运行这玩意,直接运行以下命令:
\begin{lstlisting}[language=bash]
    cd hello # 进入hello目录
    xmake # 构建
    xmake run # 运行
    xmake run -d # 调试运行
\end{lstlisting}

\subsection{语法速通}
\subsubsection{最小示例}
\begin{lstlisting}
-- xmake.lua
set_project("hello")
set_version("1.0.0")
add_rules("mode.debug", "mode.release")   -- 自动生成 debug/release 配置

target("app")                             -- 一个目标
    set_kind("binary")                    -- 可执行文件
    add_files("src/*.cpp")                -- 通配符
    set_languages("c++20")                -- 标准
    add_packages("fmt")                   -- 依赖 fmt 头文件+库
\end{lstlisting}
我们发现,对比CMake,XMake的语法更加简洁,没有复杂的概念;无需手写类似于\texttt{if (WIN32)}这样的系统条件判断,XMake会自动根据宿主处理;虽然也有目标的概念(\texttt{target}块),但是这一个块就相当于CMake的\texttt{add\_executable}(或者\texttt{add\_library})和\texttt{target\_*}的组合。

在XMake中,目标也有作用域的概念。不过写起来非常简单,例如\texttt{add\_includedirs("include", {public = true})}表示这个头文件目录是公共的,链接到这个目标的其他目标也可以使用这个头文件目录。

\subsubsection{语法速查(和CMake对比)}

\begin{table}
\centering
\begin{tabular}{l|l|l}
\hline
CMake & XMake & 说明 \\
\hline
\texttt{add\_executable} & \texttt{target("app")}, \texttt{set\_kind("binary")} & 声明一个可执行文件或库 \\
\texttt{add\_library} & \texttt{target("lib")}, \texttt{set\_kind("static")} & 声明一个静态库或动态库 \\
\texttt{target\_sources} & \texttt{add\_files("src/\*.cpp")} & 添加源文件 \\
\texttt{target\_include\_directories} & \texttt{add\_includedirs("include")} & 添加头文件目录 \\
\texttt{target\_compile\_definitions} & \texttt{add\_defines("FOO")} & 添加编译选项 \\
\texttt{target\_compile\_options} & \texttt{set\_cxflags("-Wall")} & 添加编译选项 \\
\texttt{target\_link\_libraries} & \texttt{add\_links("bar")} & 添加链接库 \\
\texttt{find\_package} & \texttt{add\_packages("fmt")} & 查找并添加包 \\
\texttt{option()} & \texttt{option("BUILD\_TESTS", true)} & 设置配置选项 \\
\texttt{install()} & \texttt{add\_installfiles("bin")} & 安装目标 \\
\hline
\end{tabular}
\caption{CMake和XMake的常用命令对比}
\end{table}

\subsubsection{目标、规则}

在XMake中,目标可以是以下种类当中的任意一种,我们可以使用\texttt{set\_kind()}来设置目标类型。关于目标和作用域的讨论,XMake和CMake如出一辙。
\begin{itemize}
    \item \texttt{binary}:可执行文件
    \item \texttt{static}:静态库
    \item \texttt{shared}:动态库
    \item \texttt{phony}:伪目标(不生成文件,只执行命令)
    \item \texttt{headeronly}:头文件库(只包含头文件,没有源文件)
\end{itemize}

XMake的规则(rules)是预定义的构建规则,可以通过\texttt{add\_rules()}来添加。我们可以理解为“编译的流水线”,例如:
\begin{lstlisting}
rule("embed")
    set_extensions(".txt")
    on_build_file(function(target, sourcefile)
        -- 把文本编译成 .o 里的字符数组
    end)
\end{lstlisting}
上述代码定义了一个名为\texttt{embed}的规则,作用是将文本文件编译成目标文件中的字符数组。XMake提供了许多内置规则,例如\texttt{mode.debug}、\texttt{mode.release}等,可以直接使用。一般情况下,多个target可以共用一个规则。

\subsubsection{调包}

XMake的包管理比CMake还要容易。我们可以非常简单地使用\texttt{add\_requires()}来添加依赖包,非常方便。

XMake的官方仓库有着九百多个常用的库。第一次编译的时候,XMake会先查本地缓存有没有需要的包;如果没有,就自动下载预编译文件或者源码。然后,XMake会自动设置include路径等必要的内容。

\begin{lstlisting}
add_requires("fmt") -- 添加 fmt 依赖
target("app")
    set_kind("binary")
    add_files("src/*.cpp")
    add_packages("fmt") -- 链接 fmt 包
\end{lstlisting}

\subsubsection{跨平台、交叉编译、远程编译}

XMake的跨平台做得很好,不需要手写toolchain文件等,只需要在命令行中指定平台和架构即可。例如:
\begin{lstlisting}[language=bash]
$ xmake f -p windows -a x64 -m release   # Windows 64-bit Release
$ xmake f -p linux -a arm64 --sdk=/opt/rpi
$ xmake
\end{lstlisting}
上述代码会在Windows上编译一个64位的Release版本,在Linux上编译一个ARM64的版本,并指定了Raspberry Pi的SDK路径。Raspberry Pi是一个流行的单板计算机(“树莓派”,常用于前端开发和嵌入式开发等轻量级场景)。

另一方面,使用XMake运行远程编译功能也很容易:
\begin{lstlisting}[language=bash]
$ xmake service --start                  # 本机当编译服务器
$ xmake f --remote_build=y
$ xmake                                  # 自动分发
\end{lstlisting}
上述代码会启动一个编译服务器,然后在本地编译时自动分发到远程服务器上进行编译。XMake会自动处理远程编译的细节,用户只需要关注代码和配置即可。

\subsection{打包发布}
XMake的打包发布也和CMake大同小异:
\begin{lstlisting}
xmake package -f deb        # 生成 .deb
xmake package -f nsis       # Windows 安装向导
xmake package -f zip        # 绿色压缩包
\end{lstlisting}

\subsection{常见坑}
XMake的常见坑和CMake类似:
\begin{itemize}
    \item 缓存出错:\texttt{xmake f -c}清理配置,比删build快。
    \item 多个版本包冲突:使用\texttt{xmake require --info <package>}查看已经缓存的包版本,使用\texttt{xmake require --extra="\{debug=true\}" <package>}强制重装。
    \item 调试脚本:使用\texttt{xmake -vD}输出详细日志。VS Code安装xmake插件也能打断点。
    \item 懒惰(不想学lua):绝大多数场景只使用6个API:target、set\_kind、add\_files、add\_includedirs、add\_links、add\_packages。其他的现查都来得及。
\end{itemize}

\subsection{从CMake迁移到XMake}
如果你已经有了一个CMake项目,想要迁移到XMake。CMake项目的结构:
\begin{lstlisting}
project/
- CMakeLists.txt
- src/
- tests/
- thirdparty/fmt/
\end{lstlisting}

首先使用cmake2xmake工具粗略转换为XMake项目:
\begin{lstlisting}[language=bash]
    xmake create -P . -t cmake2xmake
\end{lstlisting}
然后手动调整生成的xmake.lua文件,主要是:
\begin{itemize}
    \item 把add\_subdirectory拆成多个target;
    \item 把FetchContent转换为add\_requires;
    \item 把CMAKE\_BUILD\_TYPE转换为set\_config("debug")或set\_config("release");
    \item 把gtest\_discover\_tests转换为add\_rules("test")等。
\end{itemize}
最后,保留旧的构建目录,并行验证新旧构建结果是否一致。

如果使用了CI等工具,需要将CI脚本中的相关命令从CMake转换为XMake。

\section{Meson}

Meson是一个开源的构建系统,主要目标是尽可能地提高开发者的生产力。它力求快速、易用,并提供最现代化的软件构建工具。Meson的设计哲学是“不挡路”(Don't get in your way),致力于让构建过程尽可能简单、快速和可靠。

与使用自定义脚本语言的CMake或使用Lua的XMake不同,Meson使用一种专门设计的、非图灵完备的领域特定语言(DSL)。这种设计的目的是为了确保构建定义的简洁性、可读性和可预测性,避免在构建脚本中出现过于复杂的逻辑。Meson的构建定义文件通常命名为\texttt{meson.build}。

Meson本身并不直接编译代码,而是作为一个元构建系统生成另一套构建系统(Ninja或Visual Studio等)的项目文件,它的默认后端是Ninja。得益于其出色的设计和性能,Meson在开源社区获得了广泛的认可,许多大型项目,如GNOME(包括较底层的GLib、GTK以及各应用程序)、GStreamer、Systemd、Mesa等,都已采用Meson作为其官方构建系统。

安装Meson非常简单,大多数系统的包管理器都提供了Meson:
\begin{lstlisting}[language=bash]
sudo apt install meson # Debian/Ubuntu
sudo pacman -S meson   # Arch Linux
pacman -S mingw-w64-ucrt-x86_64-meson # MSYS2 on Windows
brew install meson     # macOS
\end{lstlisting}

Android的Termux等少数环境没有提供Meson,由于Meson是Python包,因此也可以方便地通过pip安装:
\begin{lstlisting}[language=bash]
pip install meson
\end{lstlisting}

\subsection{最小运行实例}
Meson强制要求“外源构建”(out-of-source builds),即构建生成的文件必须位于一个与源码目录分开的独立目录中,保持源码树的整洁。

假设我们有一个简单的\texttt{main.c}文件。在同级目录下,我们创建一个\texttt{meson.build}文件:
\begin{lstlisting}
# meson.build
project('hello_meson', ['c'],
  version : '0.1',
  default_options : ['c_std=c11'])

executable('hello', 'main.c')
\end{lstlisting}

然后,我们按以下步骤进行构建:
\begin{lstlisting}[language=bash]
# 配置项目,创建构建目录 builddir
# 可以通过`--buildtype` 来设置构建类型
meson setup builddir --buildtype=debugoptimized

# 编译
# -C 指定构建目录
meson compile -C builddir

# 运行
./builddir/hello
\end{lstlisting}

整个过程清晰明了。\texttt{meson setup} 步骤只会执行一次(除非构建脚本或环境变化),后续的修改只需要运行\texttt{meson compile -C builddir}。其中,\texttt{--buildtype}选项可以指定构建类型,如\texttt{debug}为调试模式,\texttt{release}为发布模式,\texttt{debugoptimized}为调试优化模式,\texttt{plain}为纯粹的编译模式(使用环境变量中的构建参数),\texttt{minsize}为最小化大小优化模式,\texttt{custom}为自定义模式。对于调试模式,Meson会自动设置编译器相关选项,添加调试符号。

\subsection{项目和目标}
每个Meson项目的根目录都必须有一个\texttt{meson.build}文件,并且第一条命令必须是\texttt{project()}。
\begin{lstlisting}
project('my_project', ['c', 'cpp']) # 项目名,使用的语言
\end{lstlisting}

目标(Target)是你想要构建的东西,可以是可执行文件或库等。
\begin{lstlisting}
# 创建一个可执行文件
exe = executable('my_app', 'main.c', 'utils.c')

# 创建一个静态库
lib = static_library('my_lib', 'lib.c', 'helper.c')

# 创建一个共享库
shared_lib = shared_library('my_shared_lib', 'shared.cpp')
\end{lstlisting}

Meson中的变量默认是不可变的,这有助于写出更可预测的构建脚本。上面代码中的\texttt{exe}和\texttt{lib}就是持有目标对象引用的变量。

\subsection{依赖项处理}
Meson的依赖项处理是其一大亮点。它提供了一个统一的\texttt{dependency()}函数来查找外部依赖。
\begin{lstlisting}
# 查找 fmt 库,如果找不到则构建失败
fmt_dep = dependency('fmt', required: true)

executable('app', 'main.cpp',
  dependencies : [fmt_dep]) # 将依赖项传递给目标
\end{lstlisting}

\texttt{dependency()} 函数会按顺序尝试多种方法来寻找依赖,包括(但不限于):
\begin{itemize}
    \item \textbf{pkg-config}:这是在Linux和类Unix系统上查找库的首选方式。
    \item \textbf{CMake}:Meson可以调用\texttt{find\_package}来查找CMake包。
    \item \textbf{内置查找器}:对于一些常用库(如Zlib、Threads),Meson有自己的查找逻辑。
\end{itemize}

\subsection{子项目与Wrap系统}
Meson拥有一个名为Wrap的内置依赖包管理器,类似于CMake的\texttt{FetchContent}。当系统上没有安装某个依赖时,Meson可以通过Wrap系统从网上下载并构建它。

要使用它,你需要在项目根目录下创建一个\texttt{subprojects}目录,并在其中放置一个\texttt{.wrap}文件。例如\texttt{subprojects/fmt.wrap}:
\begin{lstlisting}[language=ini]
[wrap-file]
directory = fmt-10.2.1
source_url = https://github.com/fmtlib/fmt/releases/download/10.2.1/fmt-10.2.1.zip
source_hash = e3b0c44298fc1c149afbf4c8996fb92427ae41e4649b934ca495991b7852b855
\end{lstlisting}
然后,在\texttt{meson.build}中,你可以像查找系统库一样查找它:
\begin{lstlisting}
# Meson会先尝试系统查找,失败后会自动在subprojects目录中寻找fmt
# fallback的第二个参数是子项目中的依赖变量名
fmt_dep = dependency('fmt', fallback: ['fmt', 'fmt_dep'])
\end{lstlisting}

\subsection{选项和配置}
你可以使用\texttt{option()}函数为你的项目定义可配置的选项。
\begin{lstlisting}
option('use_feature_x', type : 'boolean', value : true, description : 'Enable feature X')
\end{lstlisting}
用户在配置时可以通过\texttt{-D}参数来修改这些选项的值:
\begin{lstlisting}[language=bash]
meson setup builddir -Duse_feature_x=false
\end{lstlisting}
要查看或修改一个已经配置好的构建目录的选项,可以使用\texttt{meson configure}:
\begin{lstlisting}[language=bash]
meson configure builddir                # 查看所有选项
meson configure builddir -Duse_feature_x=true # 修改选项
\end{lstlisting}

\subsection{交叉编译}
交叉编译是Meson的一等公民。所有与交叉编译相关的设置都集中在一个单独的“交叉文件”(\texttt{cross file})中。一个简单的交叉编译文件示例如下:
\begin{lstlisting}[language=ini]
[binaries]
c = 'loongarch64-linux-gnu-gcc'
cpp = 'loongarch64-linux-gnu-g++'
ar = 'loongarch64-linux-gnu-ar'
strip = 'loongarch64-linux-gnu-strip'

[host_machine]
system = 'linux'
cpu_family = 'loongarch64'
cpu = 'loongarch64'
endian = 'little'
\end{lstlisting}

在配置时使用\texttt{--cross-file}参数指定此文件:
\begin{lstlisting}[language=bash]
meson setup build_loongarch64 --cross-file loongarch64-linux-gnu.txt
meson compile -C build_loongarch64
\end{lstlisting}
Meson会处理好所有细节,确保使用正确的编译器和库路径。

\subsection{特点与比较}
\begin{itemize}
    \item \textbf{语法和设计哲学}:Meson的语法是声明式的、非图灵完备的DSL,虽然这造成了一些限制,但这也使得构建脚本非常简洁且易于分析。它强制实施了许多最佳实践(如外源构建)。
    \item \textbf{性能}:Meson的配置阶段通常非常快。由于其默认后端是Ninja,编译速度也极具竞争力。
    \item \textbf{易用性}:对于新项目或初学者来说,Meson和XMake的上手难度通常低于CMake。Meson的错误提示非常友好,文档也极为出色。
\end{itemize}

\subsection{技巧与提示}
\begin{itemize}
    \item \textbf{修改配置}:有时候因为Meson版本升级或者我们自己需要修改配置中的内容,需要重新配置项目构建目录。此时不必手动删除\texttt{builddir}目录来重新配置,使用\texttt{meson setup --reconfigure builddir}。
    \item \textbf{调试}:Meson的输出信息非常清晰。如果遇到问题,仔细阅读它的错误提示与引导通常就能解决问题。
    \item \textbf{官方文档}:Meson的\href{https://mesonbuild.com/}{官方文档}是学习和解决问题的极佳资源,内容详尽且组织良好。
\end{itemize}

Meson提供了一种不同的构建体验。如果你正在开始一个新的开源项目,Meson是一个非常值得考虑的优秀选择。

\end{document}
