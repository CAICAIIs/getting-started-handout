\documentclass[../main.tex]{subfiles}

\begin{document}
\chapter{购买计算机}

\begin{flushright}
    \emph{本章可能含有消费建议,可能已经过时。}
\end{flushright}

如你所见,本章将介绍如何购买一台计算机。

与其他章节不同,这一章我想了很久很久才动笔,因为我知道,计算机的购买涉及到很多方面的知识和经验。同时,我也有可能因为个人的偏见和不充分的经验,给出一些不够全面或不够客观的建议,因此引发部分人的反感。所以,我预先在此声明:\textbf{本章内容仅供参考。主观上本章并不含有也不会含有任何商业广告。}

\section{获取计算机的途径}

一般情况下,我们有两种情况获取一台计算机:要么直接购买整机(笔记本或者整机台式机),要么购买零部件组装一台计算机(往往是台式机)。前者的优点是简单方便,缺点是性价比低;后者的优点是性价比高且高度可定制,缺点是需要一定的组装技术与经验,并对计算机的基本硬件知识有一定的了解。

其实希望购买零配件组装整机的人往往是对计算机性能提出更高要求的人,尤其是在游戏、图形设计、视频编辑等领域。因此,实际上组装机往往确实价格上更贵。

\section{买机器的原则}

\begin{itemize}
    \item \textbf{买新不买旧}:计算机的更新换代非常快,旧机器的性能往往无法满足新软件的需求,甚至可能无法运行新操作系统。旧机器的硬件也可能存在兼容性问题,导致无法使用最新的软件和驱动程序。一般只考虑近两年上架的产品。
    \item \textbf{确定型号和参数}:在购买之前,我们非常建议先确定好型号和参数,不能使用任何描述性的语言来作为购买依据,例如“Intel 12代高性能处理器”是描述性的,实际上对应的型号可能是 i7-12700H或者N5095,这两个玩意性能差距有七倍。
\end{itemize}

\subsection{笔记本电脑的简单分类}

笔记本电脑可以简单分类为以下几类:

\begin{itemize}
    \item \textbf{轻薄本}:轻薄本是指重量轻、厚度薄的笔记本电脑,通常用于日常办公、学习和娱乐。它们通常配备低功耗处理器,续航时间较长,但性能相对较弱。
    \item \textbf{游戏本}:游戏本是指专门为游戏设计的笔记本电脑,通常配备高性能处理器和独立显卡,能够运行大型游戏和图形密集型应用。它们通常较重,续航时间较短,但性能强大。很多人反映,背着这玩意去教室困难,请谨慎选择。
    \item \textbf{全能本}:全能本是指兼顾轻薄和性能的笔记本电脑,通常配备中高功耗处理器和独立显卡,能够满足日常办公、学习和娱乐的需求,同时也具备一定的游戏性能。
\end{itemize}

\subsubsection{MacBook系列}
由于苹果系列产品的特殊性,这里单独介绍。

MacBook在保持了轻薄的同时,拥有动辄十小时的超长续航和较强的性能。在习惯了 macOS 操作逻辑之后,使用MacBook会获得极流畅舒服的体验,如果你恰好拥有其他苹果设备构成生态,也会使得工作效率大幅提升。作为类Unix系统,macOS在编程开发时配置环境较为容易。对于音视频编辑的工作,MacBook也有较大优势。

对于学生党来说,苹果最大的缺点其实是贵,同时,游戏体验一般。且少数特定软件在苹果的 MacOS 上支持不佳,因此选购MacBook前务必向学长学姐打听好软件支持。就统计来看,各种专业的绝大多数必备软件都是支持的,个别研究方向可能出现此问题。就信科来看,计算机专业的软件几乎都有 macOS支持,电子专业则有部分软件不能运行在MacBook上。

推荐选择苹果官网作为Macbook的购买渠道,除了可以自定义配置以外,进行学生认证后可以获得优惠、并有礼品赠送(通常是耳机),价格几乎是全网最低。而且,即使是拆封激活后也可以可以七天无条件免费退货,有购物保障。

选购 Macbook 时主要的定制参数就是内存和硬盘,这里就涉及到苹果最大的问题:内存和硬盘很贵,由于内存和硬盘都无法扩展,建议至少选配 16GB 统一内存和 512GB 固态硬盘。如果提高配置之后预算超过上限,建议选购Windows本。

同时,M1、M2芯片的MacBook Air/Pro均仅支持至多一块外接显示器,只有M1 Pro/Max、M2 Pro/Max才支持多块屏幕,如有相关需求需要在选购时注意。

\subsection{奸商常见套路}

奸商常见套路有以下几种:
\begin{itemize}
    \item 模糊配置:没有写明具体配置,尤其是采用“描述性语言”而不是具体型号,消费者完全无法判断性能与实际价格。
    \item 偷梁换柱:跟你说是配置A的电脑,实际卖给你的是配置B的电脑,消费者不知不觉就上了套。
    \item 突然缺货:等你咨询完准备下单之后,突然跟你说你要的那款没货,然后让你换成所谓同等价位实际却差很远的电脑。
\end{itemize}

\subsection{专业测评}

这里为大家推荐一个较为专业客观的公众号:\textbf{笔吧测评室}。

\section{整机}

\subsection{品牌的选择}

购买整机首先要面对的是品牌。联想、戴尔、惠普、华硕、宏碁、苹果、微软 Surface、华为、小米、荣耀、雷蛇、微星、技嘉、神舟、机械革命、机械师、雷神、炫龙、火影、吾空、未来人类……名字多得像超市货架上的零食。

一般有以下两个思路:

\begin{itemize}
    \item \textbf{只看御三/四家}:联想、戴尔、惠普。它们的产品线丰富,覆盖了从入门级到高端级的各个价位段,售后服务也相对完善。以上三家市场占有率高,售后网点多,配套驱动更新及时且长期维护,虽然贵一些,但是适合不想折腾的人。苹果近年来逐渐加入了这个行列,虽然价格更贵且macOS兼容性较差,但是对于设计师和创作者来说,macOS是一个很大的优势。
    \item \textbf{只看性价比}:神舟、机械革命、火影、吾空、未来人类,同配置常常比御三家便宜一两千,但售后依赖返厂,品控如同抽盲盒。
\end{itemize}

\subsection{渠道:线上和线下}

\subsubsection{线上}

线上购买渠道不少,主要有这四种:京东、天猫、官网、拼多多百亿补贴。一般认为,京东自营大于天猫旗舰店,约等于官网,大于拼多多百亿补贴。

百亿补贴便宜是真便宜,翻车是真翻车,水深得很。要是贸然入坑,一定要做好功课,到货以后也要录开箱视频+查SN+七天无理由退货。

\subsubsection{线下}

\textbf{如果你确实需要这本手册,那么我非常不建议你去线下任何门店购买任何计算机!}

线下主要有品牌直营店、授权专卖店、电脑城、商超等。一般前两个渠道售后服务较好,但是价格往往比线上购买高一些,好处是能够当场验机并激活常用软件等。

电脑城水最深,包括并不限于转型机、展示机、矿机翻新、贴标内存,防不胜防。新手极不建议去趟浑水(可以去试试手感,但是不要买;熟人带路也不能买,\textbf{坑的就是熟人}!)。

\subsection{验机、保修}

无论线上还是线下,拿到机器后务必“开箱-插电-不联网”三步走:
\begin{itemize}
    \item 开箱:检查外包装是否完好,是否有明显的磕碰、划痕等。
    \item 插电:插上电源,检查电源适配器是否正常工作,电源指示灯是否亮起,电池是否充电。不要先进系统,进BIOS检查硬盘通电次数和电池循环次数,一般小于十次是可以接受的。
    \item 不联网:不要联网,也不要激活系统和软件;先检查屏幕是否有坏点、漏光、色偏等问题。可以使用U盘拷入DisplayX、AIDA64等工具进行检测。有部分品牌在激活Office或联网后就不给无理由退换了,所以尽可能检查清楚。
\end{itemize}

\textit{Win11新机器默认情况下需要注册微软账户,不联网无法完成系统初始化。这时,可以在该初始化界面按下\texttt{Shift+F10}打开命令行,输入\texttt{OOBE\textbackslash BYPASSNRO},然后重启电脑。这样就可以跳过联网注册微软账户的步骤。}

发现问题尽早退换货,防止不必要的麻烦。

保修政策要看清:
\begin{itemize}
    \item 全国联保 $\neq$ 全球联保。留学生买美版 ThinkPad 回国,坏了要送香港修。
    \item 上门服务 $\neq$ 免费上门服务。戴尔 ProSupport 可以第二天上门,但那是你多花 800 块买的。
    \item 意外险 $\neq$ 全保。液体泼溅、跌落、电涌,有的意外险只赔一次,第二次自费。
\end{itemize}

\section{组装机}

组装机小白有一个误区:先选CPU再选别的。这并不合适。因为 CPU 的选择会影响主板的选择,而主板又会影响内存、显卡等其他部件的选择。因此,建议先确定自己的需求,再根据需求来选择合适的 CPU、主板、内存、显卡等。

正确的顺序是:\textbf{先定需求,再定预算;先买核心配件(CPU、主板、内存、显卡),再买其他设备}。

定需求:
\begin{itemize}
    \item \textbf{办公学习}:i3-14100 / R5-5600G + 核显。价格约 5000 元。
    \item \textbf{主流网游 / 轻度剪辑}:i5-14400F / R5-7500F + RTX 4060 / RX 7600。价格约 8000 元。
    \item \textbf{2K 高刷 / 3A 大作}:i7-14700KF / R7-7800X3D + RTX 4070 Ti SUPER。价格约 12000 元。
    \item \textbf{生产力 / 4K 剪辑 / AI}:i9-14900K / R9-7950X3D + RTX 4080 SUPER 或 Ada 工作站卡。价格约 20000 元。
\end{itemize}

低于5000元的配置不建议购买组装机,建议用这个钱购买整机,需求肯定够了。

除了以上推荐的配置以外,我们还有其他需要注意的地方,例如需要做数据处理的同学们应该需要巨大的内存等。

然后定机箱体积,这个事实上决定了你机器的占地体积。这需要根据自己的桌子空间来定:全塔、中塔、MATX、ITX。ITX 溢价高,风道难做,慎入。

\subsection{核心配件}

\subsubsection{板U套装}

买板U套装一般便宜,且主板和 CPU 的兼容性有保障。如果不买板U套装,可以买散片CPU,比盒装便宜得多。当然,散片风险也大一些。

目前电脑主流 CPU 主要有 Intel 和 AMD两种,当前两者芯片势均力敌,都可以放心选择。Intel名声较大,多年以来在市场占有率上有着巨大的优势;AMD在2017 年提出新架构以后,性能突飞猛进,“AMD性能差” 已成为错误认知。

\begin{itemize}
    \item Intel 14代:接口LGA 1700;主板B660/B670/Z790。注意区分内存插槽类型,DDR4和DDR5的主板不兼容。
    \item AMD 7000系列:接口AM5;主板B650/A620/X670。A620不超频,适合游戏党。
    \item 老东西:Intel 12 i5-12490F + B660仍然是性价比之王。
\end{itemize}

\subsubsection{显卡}

2025的显卡市场依然混乱,预算充足的情况下无脑英伟达40系列即可。AMD性价比高得多,但是光追和生产力依然不如英伟达。Intel Arc驱动终于正常了,剪视频很不错,游戏党建议观望。

避坑:二手显卡风险巨大,30系默认矿卡,40系默认锁算力。(除非你认识买家!)另一方面,淘宝上所谓的“电竞显卡”店往往是丐版贴牌,慎入。

\subsubsection{内存和存储}

内存方面,DDR4和DDR5的主板不兼容,必须注意。

25年一般喜欢DDR5,盯着32GB 6000MHz CL30的套装买就可以了。英特尔14代可以冲7200MHz。(其实能做到这么好的厂家也就海力士了)

DDR4已经白菜价了,对于老东西而言性价比超级高。

存储方面,SSD的速度和容量是关键。QLC便宜,但是掉速严重。系统盘建议TLC,仓库盘可以选择QLC。机械硬盘除非需要超大容量(4TB以上),否则不建议买,一定要买的话盯着西数和希捷就行。叠瓦盘别碰。

\subsection{其他配件}

作者在这里踩过坑,总结出来的经验是:一定不能因为是非核心配件就只图便宜,盯着大牌子买总没问题。

\subsubsection{电源}

首先务必确定以上你买了些什么。一般的,瓦数至少是核心配件瓦数的1.5倍再加100瓦。盯着金牌全模组电源买就行,航嘉、长城、海韵、振华都可以。

\subsubsection{散热}

i5/R5 单塔 4 热管(利民 PA120)即可;i7/R7 双塔 6 热管(利民 FC140);i9/R9 360 水冷起步(瓦尔基里 A360)。

\subsubsection{机箱}

这个没什么可说的,一般对于高性能计算机,前进风至少3风扇,后出风至少1风扇,侧面风扇和顶风扇可选。玻璃侧透谨慎购买,尤其是闷罐,可能导致箱内温度极高。

兼容性方面,显卡长度、CPU 散热器高度、主板尺寸(ATX/MATX/ITX)、电源长度等都要考虑。

\subsection{显示器}

显示器是一种输入输出设备,按理说应该属于“其他配件”,但是我还是单独把他拿出来讲了。这是因为,显示器\textbf{对使用者而言}完全就是整个计算机中最重要的部分之一,毕竟这玩意能真正直接影响到你的所有使用体验:一个糟糕的显示器完全能够抵消一块4090带来的快乐!

\subsubsection{尺寸与分辨率}

一般来说,显示器的尺寸以对角线长度来表示,单位是英寸。分辨率则是指屏幕上像素点的数量,常见的有 1080p、2K、4K 等。为了定义屏幕的清晰度,我们引入一个概念:PPI(Pixels Per Inch),即每英寸的像素点数。PPI 越高,屏幕越清晰。具体的公式这里不啰嗦了,我们可以使用这个\href{https://config.net.cn/tools/PixelToPpi.html}{工具}来计算 PPI。

\begin{table}[ht]
\centering
\begin{tabular}{l|l|l|l|l}
\hline
对角线 & 分辨率 & 近似 PPI & 推荐视距 & 典型用途 \\
\hline
24英寸 & 1920$\times$1080 & 92 & 60–70 cm & 办公、网课、MOBA \\
27英寸 & 2560$\times$1440 & 109 & 65–75 cm & 全能甜点 \\
27英寸 & 3840$\times$2160 & 163 & 55–65 cm & 代码、设计、4K 影音 \\
32英寸 & 3840$\times$2160 & 138 & 70–80 cm & 剪辑、影视后期 \\
34英寸 & 3440$\times$1440 & 110 & 65–75 cm & 带鱼屏游戏、股票 \\
\hline
\end{tabular}
\end{table}


经验法则:1080p 别超过 24英寸,否则PPI太低,内容将非常模糊,对眼睛完全就是一种折磨。27英寸屏幕起步 2K;4K 最好 27–32英寸,否则缩放比例尴尬。带鱼屏(21:9)优先 3440$\times$1440,2560$\times$1080 纵向 PPI 太低。

\subsubsection{刷新率}

刷新率指的是显示器每秒钟能够更新的画面数量,单位是赫兹(Hz)。一般的,人类看到的显示屏至少得90Hz以上,看起来才足够舒服。

\begin{itemize}
    \item 60 Hz:办公、影音足够,不过现在已经很低端了。
    \item 75–100 Hz:轻度电竞、日常使用,预算友好,且非常流畅。
    \item 144–165 Hz:FPS 玩家黄金档,显卡吃到 RTX 4060 以上即可跑满。
    \item 240 Hz 及以上:CS2、APEX 职业选手专属,钱包与显卡双重考验。
\end{itemize}

注意:高刷必须搭配 DP1.4 或 HDMI 2.1 线,否则 1080p 240 Hz 只能缩到 144 Hz。

\subsubsection{面板技术}

面板技术指的是显示器使用什么类型的材料进行显示。主要有以下几种,其中IPS、VA、TN 是液晶面板技术,OLED 是有机发光二极管技术,Mini-LED 是一种新型的背光技术。

\begin{table}[ht]
\centering
\begin{tabular}{l|l|l|l}
\hline
面板 & 优点 & 缺点 & 适合人群 \\
\hline
IPS & 颜色准、可视角度大 & 普遍漏光 & 设计、办公、全能 \\
Fast IPS & 1 ms GTG、高刷 & 对比度一般 & 电竞、兼顾创作 \\
VA & 高对比度、色彩艳 & 响应慢、拖影 & 影音、单机 3A \\
TN & 极快响应、便宜 & 可视角度渣 & 纯 FPS 硬核玩家 \\
OLED & 无限对比、极快响应 & 烧屏风险、贵 & 影音发烧、HDR 游戏 \\
Mini-LED & 高亮度、多分区背光 & 光晕、价高 & HDR 剪辑、次旗舰电竞 \\
\hline
\end{tabular}
\end{table}

避坑提醒:

“IPS 级别”$\neq$ IPS,可能是廉价 AHVA。

VA 曲面屏 27英寸 以下意义不大,32英寸 以上才显沉浸。

OLED 桌面静态 UI 易烧屏,建议隐藏任务栏 + 黑色壁纸。

\subsubsection{色彩与亮度}

色域、色准一般用户基本上不需要考虑。对于板绘画师、视频剪辑师等专业用户来说,色域和色准则是非常重要的,这边更推荐有相关爱好的同学们咨询业内大佬,我就不在这里班门弄斧了。

亮度比较重要,尤其是对画面有追求的用户而言,带HDR的显示器几乎是必备。一般SDR 250 nit 起步,HDR400 认证只是“能亮”;真想 HDR 观影选 HDR600 以上 + 分区背光或 OLED。

\subsubsection{接口与线材}

\begin{itemize}
    \item DP1.4:现代主流接口,支持 2K 165 Hz / 4K 144 Hz 10bit 无压缩。
    \item HDMI 2.1:主机党接 PS5/XSX 4K 120 Hz 必须。
    \item Type-C:65–90 W 反向供电 + DP Alt-mode,笔记本外接一条线搞定。
    \item USB-B 上行口:老式 KVM 或显示器集成 USB Hub 时才会用到。
\end{itemize}

线材别图便宜:

2K 165 Hz 以上务必买 VESA 认证 DP1.4 线(10-20 元杂线会黑屏);HDMI 2.1 认准超高速认证(48 Gbps);Type-C 线看 E-Marker 芯片,标 100 W 却只支持 60 Hz 的比比皆是。

\subsubsection{验屏}

收到屏幕之后,建议对屏幕进行以下检查以规避问题。坏点指的是显示器上无法显示的像素点,漏光指的是屏幕边缘或角落出现的光线泄漏现象。对于需求较高的用户而言,也请验证刷新率和色域。

\begin{itemize}
    \item 坏点:使用纯色图片,红绿蓝白黑共五张,肉眼距离 50 cm 观察。国标允许 3 个以内坏点,超过即退换。
    \item 漏光:全黑图,手机夜景模式拍照,四角漏光若呈“黄雾”可接受,“白雾”则太严重。
    \item 刷新率、色域:\href{https://www.testufo.com/}{UFO Test} 网站跑 144/165/240 Hz,看是否掉帧;DisplayCAL 校色仪验证色域覆盖与 $\Delta E$。
\end{itemize}

注意:长时间盯着UFO Test网站可能导致眩晕,建议晕车的同学谨慎使用。

至此,显示器选购的坑与雷已悉数奉上。记住一句话:屏幕是你每天盯得最久的部件,预算再紧也别在这上面省过头。

\section{捡垃圾}

我们非常不推荐购买任何二手计算机,因为二手计算机的风险极大。如果确实预算有限,或者仅仅是想要体验组装电脑的乐趣,可以考虑捡垃圾(二手)。捡垃圾有两种方式:一是从身边的朋友那里获取二手设备,二是从网上的二手市场获取。对于这一方面,我建议同学们查看较早一些的教程。

捡垃圾也有一定的底线:
\begin{itemize}
    \item 电源永远不捡垃圾:电源一炸,整个机器全完。
    \item 机械硬盘永远不捡垃圾:SMART重映射扇区大于100可以直接扔了。
    \item 当面交易优先:二手市场水极深,CPU-Z、GPU-Z、HWiNFO、CrystalDiskInfo等工具都要用起来。
\end{itemize}

\end{document}