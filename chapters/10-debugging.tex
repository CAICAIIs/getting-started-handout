\documentclass[../main.tex]{subfiles}
\begin{document}

\chapter{调试和测试}

我们在前面的章节中已经知道,代码中出现错误是难免的事情。无论是语法错误、语义错误,还是不能称得上错误但是不符合预期的行为,我们都需要进行调试和测试来找出问题所在。

一些大公司专门有“测试工程师”这类岗位来负责代码正式上线之前的测试工作,以检查代码的正确性和可靠性;但是对于大多数小型项目而言,这些工作往往是由开发人员自己完成的。在本章节中,我们会介绍一些常用的调试和测试方法,帮助同学们更好地理解和解决代码中的问题。

总得说来,调试和测试就像诊治一个危重病人(值得庆幸的是,这个病人能够复活)一样。此时,我们需要四步走:先救命、再治病、再调养,最后购买医疗保险。顺序不要乱,一步都不能少。

\section{先救命}

最常见的情况是:代码崩溃了,程序完全无法执行。对于Python而言,它的报错信息非常详细,通常可以直接定位到出错的行数和错误类型,因此往往只需要根据报错信息进行修复即可。

而对于C/C++这类语言而言,它们是静态的,运行时错误能过编译,而且它们的报错信息通常非常简洁,仅凭报错信息很难定位到具体的错误地点。因此,我们需要使用一些调试工具来帮助我们定位错误。

\emph{什么,你问我怎么确定编译错误?我认为编译错误应该由编译器来判断,而不是我们来判断;编译过一次以后,你的代码编辑器应该也能够显示哪里有编译错误!}

\subsection{使用调试器}

我们在开发的过程中可以使用VS Code调试器,但是有时候我们无法获取程序源码。在这种情况下,我建议同学们使用那个最经典的调试工具——\texttt{gdb}。它是GNU项目的一部分,支持多种编程语言,包括C、C++、Fortran等。当然对于一些会读汇编的同学们而言,我觉得可以用objdump这类反汇编器来反汇编程序,并对这些汇编代码进行阅读;常见的反汇编器还有著名的IDA Pro,它是一个商业软件,价格很贵,但是功能绝对对得起它的价格:它甚至能够对反汇编出来的东西进行自动分析!
\begin{lstlisting}[language=bash]
    objdump -d ./your_program > asm.s
\end{lstlisting}

当然大多数人是没这个能力也没这个毅力去读汇编的,因此GDB最终还是我们最常用的动态调试工具。例如,以下命令可以启动GDB并加载程序:
\begin{lstlisting}[language=bash]
    gdb ./your_program
\end{lstlisting}

在GDB中,我们可以使用以下命令来设置断点、运行程序、查看变量等:
\begin{itemize}
    \item \texttt{break}:设置断点,可以简写为\texttt{b},例如\texttt{break main}在\texttt{main}函数处设置断点。也可以在特定的某一行设置断点,例如\texttt{break 42}在第42行设置断点;也可以利用偏移量来设置断点,例如\texttt{break +10}在当前行的往前数10行设置断点。
    \item \texttt{run}:运行程序。可以简写为\texttt{r}。
    \item \texttt{next}:执行下一行代码,可以简写为\texttt{n}。如果当前行是函数调用,则不会进入函数内部,而是把这个函数视为一个整体往下执行一行。该命令有一个变体\texttt{ni},它是汇编级别的断点定位,也就是执行下一条汇编指令。
    \item \texttt{step}:也是执行下一行代码,可以简写为\texttt{s}。如果当前行是函数调用,则会进入函数内部,逐行执行函数内部的代码。\texttt{si}是它的变体,表示汇编级别的单步执行。
    \item \texttt{continue}:继续运行直到下一个断点,可以简写为\texttt{c}。
    \item \texttt{print}:打印变量的值,例如\texttt{print variable},可以简写为\texttt{p variable}。如果变量是一个结构体或类的实例,可以使用\texttt{print variable.field}来打印某个字段的值。
    \item \texttt{backtrace}:查看函数调用栈,可以简写为\texttt{bt}。这对于定位程序崩溃时的调用路径非常有用。
    \item \texttt{layout}:切换到图形界面模式,可以使用\texttt{layout src}来显示源代码,使用\texttt{layout asm}来显示汇编代码。
    \item \texttt{info}:查看程序的状态,例如\texttt{info breakpoints}查看断点信息,\texttt{info registers}查看寄存器状态,\texttt{info locals}查看局部变量等。
    \item \texttt{watch}:设置观察点,当某个变量的值发生变化时暂停执行,例如\texttt{watch variable}。这对于调试复杂的逻辑错误非常有用。
    \item \texttt{set}:强制设置变量的值,例如\texttt{set variable = value}。这对于调试时修改变量的值非常有用。
    \item \texttt{list}:查看源代码(带行号),可以简写为\texttt{l}。
    \item \texttt{quit}:退出GDB。
\end{itemize}

当然,如果想要显示行号的话,我们编译代码的时候需要加上\texttt{-g}选项,例如:
\begin{lstlisting}[language=bash]
    gcc -g -o your_program your_program.c
\end{lstlisting}
也只有加上了\texttt{-g}选项,GDB才能够显示源代码和行号,\texttt{s}和\texttt{n}命令才能够逐行执行源代码,但是\texttt{si}和\texttt{ni}命令仍然能够正常执行。

以上命令其实很复杂,需要同学们多加练习才能熟练掌握。这里我推荐一个GDB的小练习:CMU ICS Lab2:BombLab。这个练习的目的是让同学们通过GDB来调试一个被加密的程序,找到正确的输入来“拆弹”。这个练习非常有趣,而且可以帮助同学们熟悉GDB的使用。



\end{document}