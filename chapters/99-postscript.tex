\documentclass[../main.tex]{subfiles}

\begin{document}

\chapter{后记}

恭喜同学们完成了本手册的阅读!

当下,人心浮躁:网文讲究的是浮光掠影,视频讲究的是短平快,已经很久没有人能够沉下心来阅读这么长的一本手册了。所以说,能看到这里的同学都是有心人,都是愿意花时间去学习、去实践的同学。你们的坚持和努力值得赞赏,也是我在缺乏合作者的时光中不断推进进度的动力。

谢谢。

不过也正常,大家看书总归是看个乐,我相信大多数人不会故意去做一些自己厌恶的事情去折磨自己。而不同的人喜欢的东西又不一样,所以说看不完手册也是非常正常的事情。毕竟人最终还是要过得快乐一些。当然,大伙都是貔貅,光进不吐,这导致整本书的内容全都是我手敲的,真是令人遗憾。(不过敲字也是我的一个爱好——这也算是因祸得福了?)

不要因为我说了这几句话就不给我提PR和Issue了啊喂 \texttt{(\#'O')} !

这份手册的前身是《计算概论衔接课》第一部分的讲义。后经过本人的思考、修改和扩充,最终形成了近百页的手册。其中,LCPU和PKUHub的同学们为我提供了许多宝贵的意见和建议,帮助我完善了手册的内容;也有许多同学在暑假课提出了问题和建议,也踩过不少坑,帮助我在编写手册时细化了许多内容、避免了许多错误。应该说,本手册编写完成,离不开众多个人与组织的无私帮助与鼎力支持。在此,也谨向所有给予我们指导、鼓励与便利的朋友们致以最诚挚的谢意。

感谢以下为本手册提供过贡献的人们(按提交时间排序):

\begin{itemize}
  \item LCPU Getting Started 全体成员
  \item PKUHub 全体成员
  \item \faGithub\href{https://github.com/wszqkzqk}{wszqkzqk}为MSYS2部分提供了极为宝贵的建议,指出了一个严重的错误
  \item \faGithub\href{https://github.com/Elkeid-me}{Elkeid-me}提供了一些常用软件的推荐
  \item \faGithub\href{https://github.com/AsTonyshment}{AsTonyshment}指出了本文的落后之处:计算中心已经启用\texttt{pku.edu.cn}域名的邮箱的二次验证和客户端专用密码功能
  \item \faGithub\href{https://github.com/ICUlizhi}{ICUlizhi}为本文推荐了现在所用的主题文件
\end{itemize}

最后,感谢每一位能够读到这里的同学。愿你们在代码与终端的世界里,既能脚踏实地,又能仰望星空;既能把系统玩得风生水起,也能把生活过得热气腾腾。

再次致谢!

\vspace{2em}
\begin{flushright}
  臧炫懿 \\
  2025年7月,在燕园
\end{flushright}

\end{document}