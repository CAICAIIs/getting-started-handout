\documentclass[lang=cn, color=blue, titlestyle=hang, scheme=chinese, 12pt]{elegantbook}

\usepackage{subfiles}
\usepackage{authblk}
\usepackage{fontawesome}
\usepackage{hyperref}
\usepackage{listings}

\setcounter{tocdepth}{2}

\title{北京大学计算机基础科学与开发手册}
\author{臧炫懿}
\institute{北京大学信息科学技术学院、北京大学学生Linux俱乐部}
\extrainfo{“把初高中失去的计算机基础知识和大学失去的开发能力补回来”}
\cover{cover.jpg}
\logo{logo.png}

\begin{document}

\maketitle

\frontmatter

\subfile{chapters/00-introduction.tex}

\mainmatter

\tableofcontents

\part{零基础起步}

\subfile{chapters/01-encounter.tex}

\subfile{chapters/02-knowledge-acquirement.tex}

\subfile{chapters/03-initial-usage.tex}

\subfile{chapters/04-buy.tex}

\part{大学计算机前置}

\subfile{chapters/05-coding.tex}

\subfile{chapters/06-text-processing.tex}

\subfile{chapters/07-drive-your-pc.tex}

\subfile{chapters/08-play-with-linux.tex}

\part{走向开发}

\subfile{chapters/09-pragmatic-coding.tex}

\subfile{chapters/10-debugging.tex}

\subfile{chapters/11-construction.tex}

\part{附录}

\appendix

\subfile{chapters/100-mini-ics.tex}

\subfile{chapters/101-cpp-guidance.tex}

\subfile{chapters/102-python-guidance.tex}

\backmatter

\subfile{chapters/99-postscript.tex}

\end{document}